
\documentclass{article}

\usepackage{amsmath}
\usepackage{amssymb}
\usepackage{amsthm}
\usepackage{graphicx}
\usepackage{multirow}

\usepackage{mathtools} % provides \xhookrightarrow

%%\usepackage{verbatim} % provides a {comment} environment for Appendix-only code
\usepackage{tikz}
\usepackage{tikz-cd}
\usetikzlibrary{shapes}

\newcommand{\N}{\mathbb{N}}
\newcommand{\R}{\mathbb{R}}
\renewcommand{\S}{\ensuremath{\mathcal{S}}} % formerly §
\newcommand{\A}{\ensuremath{\mathcal{A}}}
\newcommand{\C}{\ensuremath{\mathcal{C}}}
\newcommand{\D}{\ensuremath{\mathcal{D}}}
\newcommand{\Q}{\ensuremath{\mathcal{Q}}}
\newcommand{\M}{\ensuremath{\mathcal{M}}}
\newcommand{\IE}{\ensuremath{\Q \xrightarrow{\;\;E\;\;} \D}}
\newcommand{\E}{\ensuremath{\widehat{E}}}
\newcommand{\delihat}{\ensuremath{\surd\E}}

\DeclareMathOperator{\dual}{op}

\DeclareMathAlphabet{\bbold}{U}{bbold}{m}{n} % to obtain Riehl's \bbold{1}

\usepackage{listings}
\lstset{
  basicstyle=\small\ttfamily,
  basewidth=0.5em,
  fontadjust=true,
  language=prolog,
  mathescape
}

\begin{document}

\newtheorem{thm}{Theorem}[section]
\newtheorem{nota}[thm]{Notation}
\newtheorem{defn}[thm]{Definition}
\newtheorem{fact}[thm]{Fact}
\newtheorem{prop}[thm]{Proposition}
\newtheorem{lemma}[thm]{Lemma}
\newtheorem{corr}[thm]{Corrolary}
\newtheorem{conj}[thm]{Conjecture}
\newtheorem{eg}[thm]{Example}

\title{DRAFT: A Categorical Formulation of Dose-Escalation Designs}
\author{David C. Norris\\Precision Methodologies, LLC\\Wayland, MA}
\date{\today{}}

\maketitle

\section{Introduction}

\begin{nota}
  The \textbf{dose levels} --- or more simply, \textbf{doses} --- of a dose-escalation trial are a strictly ascending finite sequence $(x_1 < ... < x_D)$ of dose intensities $x_d \in \R^+$.  The formulation advanced here refers to these doses by their indices $\{1,...,D\}$, preserving their order but abstracting away from their numerical magnitudes.\footnote{But in what follows, it will be seen that a few general premises about the quantitative spacing of the dose levels will be important for justifying certain heuristics.}
\end{nota}

\begin{nota}
  The \textbf{participants} in a dose-escalation trial, indexed by $i \in I$, \textbf{enroll} at time $t_0^i$ into dose $d^i$.  Given that toxic responses generally manifest with some latency after dose administration, \textbf{toxicity assessment} remains \textbf{pending} for participant $i$ until some time $t_1^i \in (t_0^i, t_0^i + \delta t]$ when the assessment \textbf{resolves} into one of three \textbf{outcomes}:
  
  \begin{itemize}
    \item Participant $i$ is found to have experienced an (intolerable) \textbf{toxicity},
    \item to have become \textbf{inevaluable} due e.g. to early withdrawal from the trial or death unrelated to toxicity,
  \item or otherwise (at $t_1^i = t_0^i + \delta t$) is assessed to have \textbf{tolerated} their dose.\footnote{Our formulation ignores late-manifesting toxicities that occur after the lapse of time $\delta t$.}
  \end{itemize}
\end{nota}

\begin{nota}
  We denote evaluability by $n^i \in \{0,1\}$ and the outcome of toxicity assessment by $y^i \in \{0,1\}$.
\end{nota}

\begin{nota}
  We write $I_d(t) \subseteq I$ for the subset of individuals enrolled at dose $d$ whose assessments have resolved by time $t$:
  $$
  I_d(t) = \{ i \in I \mid d^i = d,\, t_1^i \le t \}.
  $$
\end{nota}

\begin{defn}
  A \textbf{dosewise tally} is an ordered pair $(t, n) \in \N \times \N$, recording the assessment of $t \le n$ toxic responses among $n$ evaluable trial participants who have received that dose.  These will be freely denoted with fraction bars as $t/n$ or $\frac{t}{n}$, or in ratio form as $t\!:\!u \equiv t/(t+u)$.  Note that $0/0$ represents a valid dosewise tally.
\end{defn}

\begin{nota}
  We will denote the set of dosewise tallies by $Q = \{ t/n \mid t, n \in \N; t \le n \}$ or $R = \{ t\!:\!u \mid t, u \in \N \}$ as needed.  We will also write elements of $Q^D$ or $R^D$ using the same notation, with context dictating that $t, n, u \in \N^D$.
\end{nota}

Observe that $Q$ equipped with a $+$ operator defined naturally,
$$
\frac{t_1}{n_1} + \frac{t_2}{n_2} = \frac{t_1+t_2}{n_1+n_2},
$$
is a commutative monoid, with identity $0/0$.  This property extends in the obvious way to $Q^D$.

\begin{defn}
  A \textbf{[full] tally} is a vector $q = (q_1, ..., q_D) \in Q^D$ giving the dosewise tally for each dose in an ascending sequence indexed by $\{1,...,D\} \subset \N$.
\end{defn}

\begin{defn}
  The \textbf{cumulative tally} is a right-continuous function of time,
  $$
  q : \R^+ \rightarrow Q^D,
  $$
  with dosewise components
  $$
  q_d(t) = \sum_{i \in I_d(t)} \frac{y^i}{n^i}.
  $$
\end{defn}

\begin{defn}
  The \textbf{pending count} is a right-continuous function of time,
  $$
  p : \R^+ \rightarrow \N^D,
  $$
  with dosewise components
  $$
  p_d(t) = | \{ i \in I \mid d^i = d,\, t_0^i \le t < t_1^i \} |.
  $$
\end{defn}

\begin{defn}
  The \textbf{state} of a dose-escalation trial is an ordered pair consisting of the cumulative tally and pending count:
  $$
  s(t) = (q(t), p(t)).
  $$
  This state sums up what is known at time $t$ about all the participants who have enrolled by then: those who proved inevaluable contribute $0/0$ to $q$; those who did or did not tolerate the drug contribute $0/1$ or $1/1$, respectively; and those whose assessments remain pending are counted in $p$.
\end{defn}

\begin{nota}
  By implicitly regarding $t$ as an arbitrary `current time' or `now', we will often freely suppress the $t$-dependence of $I_d$, $q$, $p$ and $s$.
  \end{nota}

\begin{nota}
  Denote by $S = Q^D \times \N^D$ the range of $s : \R^+ \rightarrow S$.
\end{nota}

\begin{nota}
  For $s \in S$, let $\underbar{s}$ denote the first component and $\bar{s}$ the second:
  $$
  s \equiv (\underbar{s},\bar{s}).
  $$
\end{nota}

\begin{defn}\label{delta-p}
  For $p \in \N^D$, we regard the \textbf{pending assessment} as the set,
  $$
  \surd p = \{ \frac{t}{n} \in Q^D \mid n \le p \} \subset Q^D,
  $$
  of all its possible resolutions.
  The surd notation $\surd$ is meant to convey the idea that $\surd p$ represents the unresolved potentiality of pending assessments that have `not yet spoken' [L. {\em surdus}, mute].  The angular appearance of $\surd$ also serves to remind of the triangular shape of each dosewise component $(\surd p)_d \subset Q \subset \N^2$.  Pronouncing $\surd$ `res' instead of `root' in this context may help.
\end{defn}

\begin{nota}
  Let $A = \surd \N^D = \{ \surd p \mid p \in \N^D \} \subset 2^{Q^D}$ denote $\surd$'s range.
\end{nota}

\begin{fact}
  For any pending assessment $\surd p$, it is possible to recover the vector $p$ from the maximal denominator:
  $$
  \bigvee \{ n \mid t/n \in \surd p \} = p.
  $$
  To see this, simply consider any resolution in $\surd p$ with all participants evaluable.
\end{fact}

\begin{fact}
  The mapping $\N^D \xrightarrow{\;\surd\;} A$ thus establishes an isomorphism:
  $$
  \N^D \cong A \subset 2^{Q^D}.
  $$
\end{fact}

\begin{fact}
  Accordingly, we may regard $S$ as the direct sum,
  $$
  S = Q^D \times \N^D \cong Q^D \oplus A,
  $$
  writing its individual elements (i.e., states) as,
  $$
  s = \underbar{s} \oplus \surd \bar{s},
  $$
  thereby conceptualizing states as \textbf{sets of possible tallies}:
  $$
  s = \{ \underbar{s} + a \mid a \in \surd \bar{s} \} \subset Q^D.
  $$
\end{fact}

\begin{nota}
  We formalize this understanding by extending $\surd$ to a function on $S$, and writing $Q^D \times \N^D = S \cong \surd S = Q^D \oplus A \subset 2^{Q^D}$.
\end{nota}

\begin{defn}
  The \textbf{plurality} of a state $s \in S$, denoted $|s|$, is defined as
  $$
  |s| = |\surd s| = |\surd\bar{s}|.
  $$
\end{defn}

\begin{defn}
  A state $s \in S$ is \textbf{singular} if $|s|=1$ and \textbf{plural} if $|s|>1$.
\end{defn}

\begin{nota}
  For any set $X$, the power set $2^X$ is customarily identified with the preorder (hence, category) $(2^X, \subseteq)$.  The opposite category $(2^X)^{\dual}$ is then the preorder $(2^X, \supseteq)$.
\end{nota}

\begin{nota}
  Let $\S = (S,\supseteq)$ denote the preorder obtained as the (full) subcategory of $(2^{Q^D})^{\dual}$ defined by $\surd$ as an inclusion functor $\S \xhookrightarrow{\;\surd\;} (2^{Q^D})^{\dual}$.
\end{nota}

By choosing to embed $\S$ in the {\em opposite} (dual) category of $2^{Q^D}$ we obtain arrows $\supseteq$ that point {\em in the direction of time}, as pending evaluations resolve and {\em information increases}.

\begin{nota}
  Let $\A = (A,\supseteq)$ denote the preorder obtained from the embedding of $A \subset 2^{Q^D}$.  Observe that $\A$ may be regarded as a subcategory of $\S$ via the inclusion functor $0 \oplus -$:
  \begin{align}
    \A & \xhookrightarrow{0 \oplus -} \S \\
    \surd p & \mapsto (\frac{0}{0},...,\frac{0}{0}) \oplus \surd p.
  \end{align}
\end{nota}

\begin{nota}
  We extend $\oplus$ to a bifunctor $\S \times \A \xrightarrow{\oplus} \S$:
  $$
  s \oplus \surd p = \underbar{s} \oplus \surd ( \bar{s} + p ),
  $$
  modeling the possibility of enrolling additional participants into a still-plural state $s$ with nonzero pending count $\bar{s}$.
\end{nota}

\begin{fact}
  Augmenting $\S$ to include the initial object $U = Q^D$ and the terminal object $\emptyset$, we obtain obtain a symmetric monoidal preorder, $\S^* = (S^*,\supseteq,U,\cap)$ with set intersection as the monoidal product and $U$ as its unit.  The essence of the proof is showing that $S^* = \surd S \cup \{\emptyset,U\}$ is closed under set intersections.  This is readily appreciated from the geometry of the components $(\surd s \cap \surd s')_d$ as intersecting isosceles right-triangular subsets of $Q \subset \N^2$.  (The formal proof may be slightly easier in the plane of $R \cong \N^2$, where the elements $t\!:\!u$ admit a symmetrical treatment of their $t$ and $u$ parts.)
\end{fact}

\begin{fact}
  The above holds true for any choice of $U \subset Q^D$, provided that we take the elements of $\S^*$ to be $\{s \cap U \mid s \in \surd S\}$.  This allows for $U$ to define a bounded set of \textbf{accessible tallies}, such as might arise from a fixed limit on trial enrollment.
\end{fact}

Note that the `null state' $\emptyset$ is a pure abstraction, unlike the {\em actual} state $\frac{0}{0} \oplus 0$ which we might well regard as obtaining upon initiation of the trial.  The `universe' $U$ is likewise an abstraction which would never obtain as an actual trial state, except in the (pathological) case where the entire planned enrollment were achieved before any assessments completed.

\section{Modeling Pharmacologic Monotonicities}

\begin{defn}
  Let $+:Q \times Q \rightarrow Q$ be defined by
  $$
  \frac{t_1}{n_1} + \frac{t_2}{n_2} = \frac{t_1 + t_2}{n_1 + n_2}.
  $$
  Observe that this is a monoidal operation with unit $0/0$, which extends in the obvious way to a monoidal operation on $Q^D$ with unit $(\frac{0}{0},...,\frac{0}{0})$.
\end{defn}

\begin{defn}\label{lesaf}
  Let $\preceq$ be the transitive closure of a preorder relation satisfying,
  \begin{equation}
  \frac{t}{n}\!+\!\frac{1}{1} \;\preceq\; \frac{t}{n} \;\preceq\; \frac{t}{n}\!+\!\frac{0}{1} \quad \forall\; \frac{t}{n} \in Q. \label{lesafcond}
  \end{equation}
  Then the preorder $(Q,\preceq)$ compares the \textbf{evident safety} expressed in dosewise tallies, such that we read
  $$
  q_1 \preceq q_2
  $$
  as ``$q_1$ is evidently no safer than $q_2$'' or ``$q_2$ is evidently at least as safe as $q_1$''.
\end{defn}

\begin{fact}
  $(Q,\preceq,\frac{0}{0},+)$ is a symmetric monoidal preorder.  It is easy to see that $+$ is a symmetric monoidal operation on $Q$ with unit $0/0$, the necessary unitality, associativity and commutativity all being inherited directly from the monoid $(\N,0,+)$.  The monotonicity condition.
  $$
  q \preceq q',\; g \preceq g' \implies q + g \preceq q' + g',
  $$
  arises by induction from the Definition~\ref{lesaf} of $\preceq$ in terms of $+$.
\end{fact}

\begin{fact}
  $$
  \frac{t}{n} \preceq \frac{t'}{n'} \quad \iff \quad t \ge t' + \max(0,n-n').
  $$
\end{fact}
\begin{proof}
  This is most easily seen by expressing \eqref{lesafcond} in its equivalent ratio form,
  $$
  t\!:\!u + 1\!:\!0 \preceq t\!:\!u \preceq t\!:\!u + 0\!:\!1 \quad \forall\; t\!:\!u \in R,
  $$
  and observing that consequently $t\!:\!u \preceq t'\!:\!u'$ iff $t \ge t'$ and $u \le u'$.  This latter condition, in turn, may be transformed as follows:
  \begin{align*}
    & t \ge t' \;\wedge\; u \le u' \\
    \iff\quad & t \ge t' \;\wedge\; n-t \le n'-t' \\
    \iff\quad & t \ge t' \;\wedge\; t \ge t' + (n-n') \\
    \iff\quad & t \ge t' + \max(0,n-n').
  \end{align*}
\end{proof}

\begin{nota}
  Let $\langle q \rangle_j$ denote the tally $(\frac{0}{0},...,\frac{0}{0},q,\frac{0}{0},...,\frac{0}{0}) \in Q^D$ with $q \in Q$ in the $j$'th position and $0/0$ elsewhere, and let $\langle q, q' \rangle_{j,k}$ denote the tally $\langle q \rangle_j + \langle q' \rangle_k$ with $q, q' \in Q$ in the $j$'th and $k$'th positions of an otherwise-$0/0$ tally.  It is to be understood that $j < k$ whenever this latter notation is used.
\end{nota}

\begin{nota}
  The sheer fact of having recorded a tally of the form $\langle\frac{1}{1},\frac{0}{1}\rangle_{j,k}$ means that we enrolled participants $i, i' \in I$ at doses $x_j < x_k$ respectively, and upon assessment found that:
  $$
  y(i,x_j) = 1, \; y(i',x_k) = 0.
  $$
  Thus we may regard $\langle\frac{1}{1},\frac{0}{1}\rangle_{j,k}$ as \textbf{equivalent to a proposition:}
  $$
  \langle\frac{1}{1},\frac{0}{1}\rangle_{j,k} \; \equiv \; \exists\, i, i' \in I \;\mbox{such that}\;\, y(i,x_j) = 1 \;\mbox{and}\;\, y(i',x_k) = 0.
  $$
  On this understanding, we can express the pharmacologic premise of \textbf{monotone dose-toxicity} via,
  $$
  \langle\frac{1}{1}\rangle_j \implies \langle\frac{1}{1}\rangle_k \;\forall\, k > j \\
  $$
  and
  $$
  \langle\frac{0}{1}\rangle_j \implies \langle\frac{0}{1}\rangle_\ell \;\forall\, \ell < j.
  $$
\end{nota}

\begin{defn}\label{domon}
  A preorder relation $\le$ on $Q^D$ is \textbf{dose-monotone} iff,
  $$
  \langle\frac{1}{1},\frac{0}{1}\rangle_{j,k} \le \langle\frac{0}{1},\frac{1}{1}\rangle_{j,k} \quad \forall \; j < k.
  $$
\end{defn}

\begin{nota}
  Let $\preceq_{ce}$ denote the binary relation on $Q^D$ obtained by componentwise extension of Definition~\ref{lesaf},
  $$
  \bigwedge_{i=1}^D (q_i \preceq q_i') \implies q \preceq_{ce} q',
  $$
  and let $\preceq_{dm}$ denote the \underline{monoidal} dose-monotonicity relation,
  \begin{equation}
  q + \langle\frac{1}{1},\frac{0}{1}\rangle_{j,k} \preceq_{dm} q + \langle\frac{0}{1},\frac{1}{1}\rangle_{j,k} \quad \forall\; j < k \quad \forall\; q \in Q^D, \label{xchg}
  \end{equation}
  which ensures that $+ : Q^D \times Q^D \rightarrow Q^D$ is monotone with respect to $\preceq_{dm}$.
\end{nota}

\begin{defn}\label{lesaf^D}
  Let $\Q = (Q^D,\preceq)$ be the free preorder generated by $\preceq_{ce} \cup \preceq_{dm}$.
\end{defn}

\begin{fact}
  Def.~\ref{lesaf^D} creates no ambiguity by its reuse of $\preceq$, since it incorporates Def.~\ref{lesaf} as the special case $D=1$.
\end{fact}

\begin{fact}\label{symon}
  $\Q = (Q^D,\preceq,\langle\frac{0}{0}\rangle,+)$ is a symmetric monoidal preorder.
\end{fact}

\begin{nota}
  Given a dose-indexed tuple $x = (x_d) \in \N^D$, let capitalization denote partial summation, such that $X_d = \sum_{j=1}^d x_j$, and $X = (X_d) \in \N^D$.
\end{nota}

\begin{fact}
  The preorder $\preceq$ is readily calculated from an equivalent condition,
  \begin{equation}
  t\!:\!u \preceq t'\!:\!u' \quad\iff\quad T \ge T' \;\wedge\; U \le U'.\label{cond}
  \end{equation}
\end{fact}
\begin{proof}
  It is the universal property of the free preorder $(Q^D, \preceq)$ that $\preceq$ is the reflexive, transitive closure of the relation $\preceq_{ce} \cup \preceq_{dm}$.  The $(\!\impliedby\!)$ direction of \eqref{cond} is immediate: the RHS obviously defines a reflexive and transitive relation on $R^D \cong Q^D$, which directly implies both $\preceq_{ce}$ and $\preceq_{dm}$ as special cases.  So the RHS of \eqref{cond} clearly defines a preorder that contains $\preceq$.

  To show the $(\!\implies\!)$ direction, consider that any statement $q \preceq q'$ with $q \neq q'$ must arise from some finite string of the form,
  \begin{equation}
  q = q^0 \preceq_* q^1 \preceq_* ... \preceq_* q^n = q' \quad n \ge 1, \label{string}
  \end{equation}
  where each $\preceq_*$ is either $\preceq_{ce}$ or $\preceq_{dm}$.  But observe that across either of these $\preceq_*$'s, both $T_d^{n-1} \ge T_d^n$ and $U_d^{n-1} \le U_d^n$ must hold $\forall d, n$.  Since the latter are themselves transitive conditions, they must hold also across the whole chain from $q$ to $q'$.  
\end{proof}

\begin{corr}\label{partord}
  $\preceq$ is in fact a \underline{partial order} on $Q^D$, since $t\!:\!u \cong t'\!:\!u'$ requires both equalities on the RHS of \eqref{cond} to hold, forcing $t_i = t_i'$ and $u_i = u_i'$ for all $\,i$.
\end{corr}

\begin{nota}
  Corrolary~\ref{partord} licenses the notation $\prec$ defined by,
  $$
  q_1 \prec q_2 \iff q_1 \preceq q_2 \;\wedge\; q_1 \neq q_2.
  $$
\end{nota}

\begin{corr}\label{lattice}
  Moreover, $\Q$ is indeed a \underline{lattice}, since it has all meets and joins, which are readily calculated from \eqref{cond}:
  \begin{equation}\label{meetjoin}
    \begin{aligned}
    q \wedge q' & \cong (t\!:\!u) \wedge (t'\!:\!u') \cong
    (T\!:\!U) \wedge (T'\!:\!U') = (T\!\vee\!T'):(U\!\wedge\!U')
    \\
    q \vee q' & \cong (t\!:\!u) \vee (t'\!:\!u') \cong
    (T\!:\!U) \vee (T'\!:\!U') = (T\!\wedge\!T'):(U\!\vee\!U').
    \end{aligned}
  \end{equation}
\end{corr}

Some additional orientation to Corrolary~\ref{lattice} can be gained as follows.  Write $\mathcal{N} = (\N^D,\le)$, and let $\M$ be the image of the partial-sums functor $\mathcal{N} \xrightarrow{\Sigma} \mathcal{N}$.  Thus, $\M \subset \mathcal{N}$ consists of the nondescending, length-$D$ sequences in $\N$ with the preorder inherited via the inclusion $\M \hookrightarrow \mathcal{N}$:
\begin{center}
  \begin{tikzcd}
    \M \ar[rr,hook]
    & & \mathcal{N}. \ar[ll,bend left,"\Sigma"]
  \end{tikzcd}
\end{center}
Now because $\M$ is closed under the meet and join operations,
$$
m, m' \in \M \implies m \wedge m', m \vee m' \in \M,
$$
$\M$ also inherits these operations from $\mathcal{N}$, along with its lattice property.  Finally, \eqref{meetjoin} shows that
$$
\Q \cong \M^{\dual} \times \M,
$$
the product of two lattices, which is also a lattice.

\subsection{Further characterization of the monoidal preorder $\preceq$}

\begin{quote}
  Make no mistake about it: Computers process numbers --- not symbols.
  We measure our understanding (and control) by the extent to which
  we can arithmetize an activity.
  
  \hfill --- Alan J. Perlis (Epigram \#65)
\end{quote}

Observe that the proof of Fact~\ref{symon} did not use the {\em symmetric monoidal} character of the $+$ operation on $\Q$.  By exploiting specifically the {\em commutativity} of $+$, however, we can obtain a more readily generalizable reformulation of \eqref{cond}.

\begin{fact}
  Every arrow of the preorder $\preceq$ may be obtained in the form,
  \begin{equation}
  \sum_{i=1}^D \lambda_i\langle\frac{1}{1}\rangle_i + \sum_{j<k} \eta_{jk}\langle\frac{1}{1},\frac{0}{1}\rangle_{j,k} +q \preceq q + \sum_{j<k} \eta_{jk}\langle\frac{0}{1},\frac{1}{1}\rangle_{j,k} + \sum_{i=1}^D \gamma_i\langle\frac{0}{1}\rangle_i, \label{clpz}
  \end{equation}
  for some $q \in Q^D$ and $\lambda_i, \eta_{jk}, \gamma_i \in \N$.  This is a CLP($\mathbb{Z}$) constraint.
\end{fact}
\begin{proof}
  It is clear that every statement of the form \eqref{clpz} is a valid arrow which we may build up by starting with the reflexive arrow $q \preceq q$, repeatedly applying $\preceq_{dm}$ (obtaining the paired $\eta_{jk}$ terms), then repeatedly applying $\preceq_{ce}$; the left-hand side of \eqref{lesafcond} generates the $\lambda_i$ terms, its right-hand side the $\gamma_i$ terms.
  The converse, that all valid arrows must be of the form \eqref{clpz}, follows from considering strings of the form \eqref{string}, and exploiting the commutativity of $+$ to collect like terms.
\end{proof}

This proof indicates how one might proceed to {\em enlarge} the basic preorder $\preceq$ while retaining its monoidality.  While any sensible evident-safety order on dose-escalation tallies probably needs to be monoidal and dose-monotone, and therefore must {\em contain} $\preceq$, these concepts do not exhaust all pharmacologic intutions one might wish to model.

\section{Dose-Escalation Protocols}

A dose-escalation protocol (DEP) is generally situated in a queueing context, where it must service the {\em arrival process} of participants presenting available for enrollment.  At any time, there may be 0, 1, or many participants waiting to enroll.  It is the task of a dose-escalation protocol to decide at what doses (if any) to enroll the waiting participants, conditional on the trial's current state.

\begin{defn}
  A \textbf{cohort} is a pair $(t,c) \in [0,\infty) \times \N^D$ giving the number of participants enrolling concurrently at time $t$ at each dose in $\{1,...,D\}$.  As previously for tallies, pending counts and trial states, a `current time' will often be implicit, so that we will freely suppress the $t$-component and speak of `cohorts' in $\N^D$.
\end{defn}

\begin{defn}\label{cohcat}
  Let $\unlhd$ denote the preorder relation formed by transitive closure of the usual preorder $\le$ on $\N^D$,
  $$
  x \le y \implies x \unlhd y \quad\forall\, x, y \in \N^D,
  $$
  together with the condition
  $$
  j \le k \implies \widehat{j} \unlhd \widehat{k} \quad\forall j,k \in \{0,...,D\}.
  $$
  Let $\C = (\N^D,\unlhd)$ denote the resulting preorder, and call its objects `cohorts'.  Observe that the arrows $c \xrightarrow{\unlhd} c'$ in this category point in the direction of increasingly `ambitious' cohorts that enroll more participants, or at higher doses.
\end{defn}

\begin{nota}
  Let $\downarrow : \C \rightarrow 2^\C$ denote the functor yielding the principal lower sets,
  $$
  \downarrow c = \{ c' \in \C \mid c' \unlhd c \}.
  $$
\end{nota}

\begin{defn}\label{deprot}
  A \textbf{rolling dose escalation [RDE]} is a functor
  $$
  \widetilde{E} : \S \rightarrow 2^\C
  $$
  that in any trial state determines a set of admissible cohorts.  The functoriality here models a caution that underlies any reasonable approach to dose escalation:
  $$
  s \supseteq s' \implies \widetilde{E} s \subseteq \widetilde{E} s',
  $$
  which is to say that in states with less information, enrollment options should be more restrictive.

  The qualifier `rolling' applies on account of the set-valued domain $\S$, meant generally to allow for `rolling enrollment' (cite Skolnik et al, 2008) even from plural states with pending assessments.
  
  The set-valued codomain of $\widetilde{E}$ is intended to allow generally for servicing waiting queues of different sizes, and even for the arbitrary exercise of `clinical judgment' in dose assignments.

  We will freely regard any functor $\widetilde{E} : \S \rightarrow \C$ as the RDE, $\downarrow\!\circ \widetilde{E} : \S \rightarrow 2^\C$.
\end{defn}

The high generality of Definition~\ref{deprot} allows for development of a taxonomy that identifies and names various desirable properties which, at their intersection, may define a class of rational DEP's of practical interest.  But we now leap ahead to a highly restrictive class, in order to make concrete progress.

\begin{nota}
  For $D \in \N^+$, let $\D$ denote the preorder consisting of the set $\{0, 1, ..., D\}$ equipped with the (reflexive and transitive) relation $\le$ defined as usual on $\N$.  In order to use categorical language, we will regard $\D$ as the category freely generated by the graph, $0 \rightarrow 1 \rightarrow \dots \rightarrow D$.\footnote{More conventionally, the ordinal category with D+1 elements would be written $\mathbb{D}+\bbold{1}$; see Riehl Example 4.1.14.  But we trust no confusion will arise from our notation $\D$.}
\end{nota}

\begin{defn}
  An \textbf{incremental enrollment [IE]} is a functor $\Q \xrightarrow{\;\;E\;\;} \D$.  Note that the functoriality here imposes the core intuition of dose-escalation,
  $$
  q \preceq q' \implies E q \le E q',
  $$
  that dose assignment should correlate with evident safety.
\end{defn}

\begin{nota}
  Let $\S \xrightarrow{\bigwedge} \Q$ denote the functor defined by the worst-possible (most toxic) resolution,
  $$
  \bigwedge s = \bigwedge_{q \in s} q.
  $$
  The right-hand side is well-defined, since any given state $s$ is finite, and $\Q$ is a lattice.  Functoriality holds because $s \supseteq s' \implies \bigwedge_{q \in s} q \preceq \bigwedge_{q \in s'} q$.
\end{nota}

\begin{nota}\label{hatfun}
  Let $\D \xrightarrow{\widehat{-}} \C$ be the functor defined by $\widehat{d} = ([j = d])_{j\in\D}$, where $[-]$ represents the Iverson bracket,
  $$
  [P] = \begin{cases*}
    1 & if $P$ is true \\
    0 & otherwise.
  \end{cases*}
  $$
  Thus, $\widehat{0}$ is a $D$-vector of all zeros, and for $d>0$, $\widehat{d}$ is the vector $(0,...,0,1,0,...,0)$ with 1 in the $d$\!'th position.  The functoriality of $\widehat{-}$ arises directly from the 2nd condition in Definition~\ref{cohcat}.
\end{nota}

\begin{nota}
  Given IE $\IE$, define $\E = \widehat{-} \circ E \circ \bigwedge$:
  \begin{center}
    \begin{tikzcd}[column sep=large,row sep=large]
      \Q \ar[r,"E"]
      & \D \ar[d,"\widehat{-}"] \\
      \S \ar[u,"\bigwedge"] \ar[r,"\E"']
      & \C
    \end{tikzcd}
  \end{center}
\end{nota}

\begin{fact}
  `$E$' and `$\bigwedge$\!' commute in the definition of $\E$:
  $$
  \E s = E \left(\bigwedge s\right) = E \bigwedge_{q \in s} q = \bigwedge_{q \in s} E q.
  $$
\end{fact}

\begin{nota}\label{altmutrec}
  Given IE $\Q \xrightarrow{\;\;E\;\;} \D$, define functors $\S \xrightarrow{\E_n} \S, n \in \N$ by,
  $$
  \begin{aligned}
    \E_0 s &= s \\
    \E_{n+1} s &= \E_n s \oplus \surd\E(\E_n s).
  \end{aligned}
  $$
\end{nota}

\begin{fact}
  Given a state $s \in \S$, the sequence $(\E_n s)_{n\in\N}$ defines a diagram in $\S$ of shape $(\N,\le)^{\dual}$:
  $$
  \dots \supseteq \E_2s \supseteq \E_1s \supseteq \E_0s\equiv s,
  $$
for which the so-called `inverse limit' is $\displaystyle{\lim_{\longleftarrow} \E_{-} s} = \bigcup_{n\in\N} \E_n s$.\footnote{Leinster [Ex. 5.1.21(d), p.120] seems to disparage the term `inverse' as ``old fashioned'', whereas Riehl [Def.~3.1.21, p.80] presents it as standard.}
\end{fact}

\begin{nota}
  Let $\E_\leftarrow : \S \rightarrow \S$ denote the functor defined by these limits,
  $$
  \E_\leftarrow s = \displaystyle{\lim_{\longleftarrow} \E_{-} s} = \bigcup_{n\in\N} \E_n s.
  $$
\end{nota}

\begin{defn}\label{rdegenbyie}
  The \textbf{RDE generated by an IE} $\IE$ is the functor $\widetilde{E} : \S \rightarrow \C$ defined by the equation,
  $$
  \E_\leftarrow s = s \oplus \surd\widetilde{E} s.
  $$
  \begin{center}
    \begin{tikzcd}[column sep=large,row sep=large]
      \S \ar[dr,"\E_\leftarrow"']
      \ar[rr,bend left,"\surd\widetilde{E}"]
      & \;\S\times \A
      \ar[l,"\pi_\S"] \ar[d,"\oplus"] \ar[r,"\pi_\A"']
      & \A \\
      & \S
    \end{tikzcd}
  \end{center}
\end{defn}

\subsection{A Concrete Construction}

In this subsection, we motivate and elucidate the rather abstract Definition~\ref{rdegenbyie} using a more concrete construction.

\begin{nota}\label{mutrec}
  Given IE $\Q \xrightarrow{\;\;E\;\;} \D$ and a tally $q \in \Q$, define sequences $e_n(q) \in \{0,...,D\}$ and $E_n(q) \in S$ inductively by the mutually recursive relations,
  $$
  \begin{aligned}
    e_1(q) &= E q \\
    E_n(q) &= q \oplus \surd \sum_{k=1}^n \widehat{e_k(q)} \\
    e_{n+1}(q) &= \bigwedge_{q' \in E_n(q)} E q'.
  \end{aligned}
  $$
This defines how, from the standpoint of a singular trial state $q \oplus \{\widehat{0}\} \in \S$, one might proceed to enroll participants from a waiting queue: enroll a first participant at dose $e_1(q)$; then, from the standpoint of the resulting plural state $E_1(q) = q \oplus \surd\widehat{e_1(q)}$, the minimax principle suggests $e_2(q) = \bigwedge_{q' \in E_1(q)} E q'$ as a suitable dose for enrolling a second participant; this yields the even larger state $E_2(q)$, which may then admit further enrollment, and so on.
\end{nota}

\begin{fact}
  The sequence $(e_n(q))_{n \in \N^+}$ is nonincreasing:
  $$
  m < n \implies e_m(q) \ge e_n(q) \quad \forall\, q \in \Q.
  $$
\end{fact}

\begin{fact}
  The sequence $(E_n(q))_{n \in \N^+}$ is nested:
  $$
  m < n \implies E_m(q) \subseteq E_n(q) \quad \forall\, q \in \Q.
  $$
\end{fact}

Except in pathological cases (\textbf{TODO:} rule these out by some explicit provision), the sequence $e_n(q)$ must become zero after a finite number of terms, at which point the sequence $E_n(q)$ converges.

\subsection{A Categorical Perspective}

The construction in Notation~\ref{mutrec} can be appreciated via cones over and under the diagram $(\N,\le)^{\dual} \xrightarrow{E_{-} s} \S$, with summit $a \in \S$ and nadir $s = q\oplus\{\widehat{0}\}$:
\begin{center}
  \begin{tikzcd}[row sep=huge]
    a \ar[d,"\supseteq" {description,rotate=-90}]
    \ar[dr,"\supseteq" {description,rotate=-45}]
    \ar[drr,"\supseteq" {description,rotate=-25}]
    \ar[drrr,"\supseteq" {description,rotate=-17}] \\
    \dots \ar[r,"\supseteq" description]
    \ar[d,"\supseteq" {description,rotate=-90}]
    & E_2s \ar[r,"\supseteq" description]
    \ar[dl,"\subseteq" {description,rotate=45}]
    & E_1s \ar[r,"\supseteq" description]
    \ar[dll,"\subseteq" {description,rotate=25}]
    & E_0s\equiv s \ar[dlll,"\subseteq" {description,rotate=17}] \\
    s
  \end{tikzcd}
\end{center}
This reveals Definition~\ref{rdegenbyie} as a limit,
$$
E^*(s) = \lim_{\longleftarrow} E_ns,
$$
the limit cone being
\begin{center}
  \begin{tikzcd}[row sep=huge]
    \displaystyle{\lim_{\longleftarrow} E_ns} \ar[d,"\supseteq" {description,rotate=-90}]
    \ar[dr,"\supseteq" {description,rotate=-45}]
    \ar[drr,"\supseteq" {description,rotate=-27}]
    \ar[drrr,"\supseteq" {description,rotate=-19}] \\
    \dots \ar[r,"\supseteq" description]
    & E_2s \ar[r,"\supseteq" description]
    & E_1s \ar[r,"\subseteq" description]
    & E_0s\equiv s,
  \end{tikzcd}
\end{center}
with projections $E^*(s) \supseteq E_ns$ defining admissible cohorts of size $n$.

\section{Extensions of standard protocols}

\begin{conj}\label{extend}
  Many standard dose-escalation protocols, including $\mbox{3 + 3}$ and BOIN designs, will extend to RDEs generated by IEs.  Solutions to this extension problem will generally be neither unique nor exact, and our approximate solutions may involve `corrections' of the original protocol.
\end{conj}

\subsection{Rectification of the $\mbox{3+3}$ trial design}

Let us examine Conjecture~\ref{extend} in the context of the $\mbox{3+3}$ trial.  The smallest nontrivial $\mbox{3+3}$ design considers $D = 2$ doses, and has 46 possible paths \cite{norris_executable_2024}.  These paths terminate in 29 distinct tallies, each with a dose-level recommendation in $\{0, 1, 2\}$ defined by the protocol.

\begin{verbatim}
:- use_module(rcpearl). % Predicates defined in Norris & Triska (2024)
:- use_module(library(lists)).
:- use_module(library(dcgs)).
:- use_module(library(lambda)).
:- use_module(library(format)).
:- use_module(library(tabling)).

:- table endtally_rec/2.
endtally_rec(FinalTally, Rec) :-
    phrase(path([0/0]-[0/0]), Path),
    phrase((..., [Endstate,stop,recommend_dose(Rec)]), Path),
    state_tallies(Endstate, FinalTally).

?- N+\(setof(Q-Rec, endtally_rec(Q, Rec), QRecs),
       maplist(portray_clause, QRecs), length(QRecs, N)).
% Output is verbatim, but reordered and tabulated for display:
%@ [2/3,0/0]-0.   %@ [0/6,2/3]-1.   %@ [0/3,0/6]-2.
%@ [2/6,0/0]-0.   %@ [0/6,2/6]-1.   %@ [0/3,1/6]-2.
%@ [2/6,2/3]-0.   %@ [0/6,3/3]-1.   %@ [1/6,0/6]-2.
%@ [2/6,2/6]-0.   %@ [0/6,3/6]-1.   %@ [1/6,1/6]-2.
%@ [2/6,3/3]-0.   %@ [0/6,4/6]-1.
%@ [2/6,3/6]-0.   %@ [1/6,2/3]-1.
%@ [2/6,4/6]-0.   %@ [1/6,2/6]-1.
%@ [3/3,0/0]-0.   %@ [1/6,3/3]-1.
%@ [3/6,0/0]-0.   %@ [1/6,3/6]-1.
%@ [3/6,2/3]-0.   %@ [1/6,4/6]-1.
%@ [3/6,2/6]-0.
%@ [3/6,3/3]-0.
%@ [3/6,3/6]-0.
%@ [3/6,4/6]-0.
%@ [4/6,0/0]-0.
%@    N = 29.
\end{verbatim}

       Thus, the $\mbox{3+3}$ trial defines a {\em partial} function $Q^2 \xrightharpoonup{F} \{0,1,2\}$ that maps a subset $|\Q_f| \subset Q^2$ of 29 final tallies to their respective dose recommendations.  In view of Conjecture~\ref{extend}, we would like to pose and solve the extension problem,
\begin{equation}\label{extension-problem}
  \begin{tikzcd}[row sep=large]
    \Q_f \ar[dr,hook,"\iota"] \ar[rr,"F"] & & \D \\
    & \Q\, \ar[ur,dashed,"E?"']
  \end{tikzcd}
\end{equation}
The subcategory inclusion $Q_f \xhookrightarrow{\iota} \Q_f$ is of course functorial, but what about $F$?  Any violation of functoriality by $F$ must take the form of final tallies $q_1, q_2 \in \Q_f$ with respective dose-level recommendations $d_1, d_2 \in \{0,1,2\}$ such that $q_1 \preceq q_2$ but $d_1 \not\le d_2$:

\begin{lstlisting}
?- endtally_rec(Q1, D1),
   endtally_rec(Q2, D2),
   Q1 '$\preceq$' Q2, % Q1 evidently no safer than Q2,
   D1 #>  D2. % yet recommended D1 exceeds D2.
%@    Q1 = [1/6,1/6], D1 = 2, Q2 = [0/6,2/6], D2 = 1
%@ ;  false.
\end{lstlisting}

Thus, interestingly, we discover that the dose recommendations of the \mbox{3+3} trial are not actually consistent with the basic pharmacologic intuition embodied in our dose-monotonicity condition of Definition~\ref{domon}: substituting $q = (\frac{0}{5},\frac{1}{5})$ into Equation~\eqref{xchg}, we see that $(\frac{1}{6},\frac{1}{6}) \preceq (\frac{0}{6},\frac{2}{6})$, yet the \mbox{3+3} design accords the {\em higher} dose to the {\em less safe} of the two tallies.  Adapting existing dose-escalation designs to the framework presented here will generally require an initial `rectification' step, in which nonmonotonicities implicit in existing designs are corrected.

\begin{lstlisting}
table mendtally_rec/2.
mendtally_rec(Q, D) :- mendtally_rec(Q, D, _).

mendtally_rec(Q, D, Ds) :-
    endtally_rec(Q, D0),
    findall(Di, (endtally_rec(Qi, Di),
                 Q $\preceq$ Qi,     % Q is no safer than Qi,
                 D0 #> Di), % yet its rec exceeds Di.
            Ds),
    foldl(clpz:min_, Ds, D0, D).

?- mendtally_rec(Q, D, [_|_]).
%@    Q = [1/6,1/6], D = 1 % the sole rectification needed
%@ ;  false.

?- mendtally_rec(Q1, D1),
   mendtally_rec(Q2, D2),
   Q1 $\preceq$ Q2,
   D1 #> D2.
%@    false. % Rectification succeeded.
\end{lstlisting}

With the aim of extracting as much information as possible from existing designs, one might suppose it useful to consider not only {\em final} tallies with their dose recommendations, but also {\em interim} tallies and their associated {\em next}-dose recommendations.  But because the latter may be entangled with considerations of trial {\em progress}, they seem less readily interpretable as expressing the design's underlying pharmacologic intuitions.

\subsection{Extending the dose-recommendation functor to an IE}

Rectification has yielded a final dose recommendation,
$$
\Q_f \xrightarrow{F} \D \equiv \{0 \le 1 \le 2\},
$$
that is {\em functorial} on the preorder of final tallies $\Q_f \subset \Q$, thereby ensuring our diagram \eqref{extension-problem} is licensed.  At first sight, \eqref{extension-problem} looks like a typical set-up for seeking a Kan extension of $F$ along the inclusion functor $\iota$:
\begin{equation}\label{Kan-nope}
  \begin{tikzcd}[row sep=large]
    \Q_f \ar[dr,hook,"\iota"'] \ar[rr,"F"{name=F},""{name=U,below}] & & \D \\
    & \Q\, \ar[ur,dashed,"\text{Lan}_\iota F"']
      \ar[Rightarrow,shorten=2.5mm,from=U,"\eta"]
  \end{tikzcd}
  \quad\quad\text{or}\quad\quad
  \begin{tikzcd}[row sep=large]
    \Q_f \ar[dr,hook,"\iota"'] \ar[rr,"F"{name=F},""{name=U,below}] & & \D \\
    & \Q\, \ar[ur,dashed,"\text{Ran}_\iota F"']
      \ar[Rightarrow,shorten=2.5mm,to=U,"\epsilon"']
  \end{tikzcd}
\end{equation}
But the components of the natural transformation $\eta$ or $\epsilon$ would then have to operate in the cramped quarters of $\D$, allowing for only the coarsest possible approximation to the desired extension.

By exchanging the roles of $\Q$ and $\D$, however, we create the opportunity for natural transformations operating in the finer granularity of $\Q$.  Consider therefore the left Kan extension of the inclusion functor $\iota$ along $F$:\footnote{See Riehl, p.190, from which these diagrams are drawn nearly verbatim.}
\begin{equation}\label{Kan-yup}
  \begin{tikzcd}[row sep=large,column sep=small]
    \Q_f \ar[dr,"F"'] \ar[rr,hook,"\iota",""{name=U,below}] & & \Q \\
    & \D\, \ar[ur,dashed,"G=\text{Lan}_F \iota"']
      \ar[Rightarrow,shorten <=2mm,shorten >=4mm,from=U,"\eta"{pos=0.4}]
  \end{tikzcd}
  \begin{tikzcd}[row sep=large,column sep=small]
    \Q_f \ar[dr,"F"'] \ar[rr,hook,"\iota",""{name=U,below}] & & \Q \\
    & \D\, \ar[ur,"G'"']
      \ar[Rightarrow,shorten <=2mm,shorten >=4mm,from=U,"\gamma"{pos=0.4}]
  \end{tikzcd}
  =
  \begin{tikzcd}[row sep=large,column sep=small]
    \Q_f \ar[dr,"F"'] \ar[rr,hook,"\iota",""{name=U,below}] & & \Q \\
    & \D\, \ar[ur,bend left,"G"{pos=0.6,inner sep=0.2},""{name=H,below}]
           \ar[ur,bend right,"G'"'{inner sep=0},""{pos=0.35,name=I,above}]
           \ar[Rightarrow,from=H,to=I,"\exists !"{pos=0.4},"\alpha"'{pos=0.6}]
    \ar[Rightarrow,shorten <=2mm,shorten >=4mm,from=U,shift right=2mm,"\eta"'{pos=0.4}]
  \end{tikzcd}
\end{equation}
in which the so-called `pasting diagram' on the right states a universal mapping property of $G = \text{Kan}_F\iota$, that any other $\D \xrightarrow{G'} \Q$ with natural transformation $\iota \xRightarrow{\gamma} G'F$ must have $\gamma$ factoring uniquely through $\eta$.

In this context specifically, the abovementioned natural transformations are simply finite collections of arrows `$\preceq$' of the preorder $\Q$, indexed either by $Q_f$ or (in the case of $\alpha$) by $\D$.  The diagrams above therefore reduce to `vectorized' statements about the images of $\Q_f$ under $\iota$ and $G F$.
\begin{equation}\label{Kan-vectorized}
  \Q_f \preceq G F \Q_f \preceq G' F \Q_f.
\end{equation}

By serially restricting \eqref{Kan-vectorized} to the nested sequence $\Q_{f\downarrow d} = F^{-1}(\downarrow\!d) \subset \Q_f$ of preimages of the lower sets of $\D$, we obtain equivalently that
\begin{equation}\label{Kan-pointwise}
  \Q_{f\downarrow d} \subseteq\; \downarrow\!G(d) \quad \forall d \in \D,
\end{equation}
the $G(d)$ being by definition the {\em lowest} such tallies for which this holds.  Now \eqref{Kan-pointwise} in turn can be translated as
\begin{equation}\label{Kan-adjointly}
  F(q) \le d \Longleftarrow q \preceq G(d),
\end{equation}
which suggests the following definition.

\begin{defn}
  A \textbf{Galois enrollment} is an IE $\IE$ for which \underline{either} a right (upper) or left (lower) adjoint exists:
  \begin{center}
    \begin{tikzcd}
      \Q \ar[rrr,"E"{name=E,description}]
      & & & \D
      \ar[lll,bend right,"G"'{name=G}]
      \arrow[phantom,from=G,to=E,"\dashv" rotate=90]
    \end{tikzcd}
  \end{center}
  or
  \begin{center}
    \begin{tikzcd}
      \Q \ar[rrr,"E"{name=E,description}]
      & & & \D
      \ar[lll,bend left,"L"{name=L,below}]
      \arrow[phantom,from=E,to=L,"\dashv" rotate=90]
    \end{tikzcd},
  \end{center}
  (An adjunction between preorders is called a Galois connection, hence the name.)
\end{defn}
Thus, in the case of a lower Galois enrollment $E \dashv G$, the upper adjoint $G$ provides the rule,
$$
E(q) \le d \quad \iff \quad q \preceq G(d),
$$
whereas for an upper Galois enrollment $L \dashv E$ we would have,
$$
d \le E(q) \quad \iff \quad L(d) \preceq q.
$$

One appeal of a Galois enrollment is that it yields a simple rule parametrized by a selection of $D+1$ tallies, and based on the easy calculation of `$\preceq$' via \eqref{cond}.  Writing $G(d) = g_d$, we have parameters $\{g_0,...,g_D\} \subset \Q$ defining a lower-Galois enrollment by partitioning $\Q$ in a bottom-up cascade: $q \preceq g_0 \implies E(q)=0$, else $q \preceq g_1 \implies E(q)=1$, and so forth.  Similarly, $\{\ell_d = L(d)\} \subset \Q$ would generate a top-down cascade defining a right-Galois enrollment.

Now if we wish to proceed {\em cautiously} in approximating some given trial with final dose recommendations $\Q_f\!\xrightarrow{F} \D$, we must ensure $E(q) \le F(q) \;\forall q \in \Q_f$.  For a lower-Galois approximation, this requires
\begin{equation}
F(q) \le d \; \Longleftarrow \; q \preceq g_d \quad \forall q \in \Q_f,\; d \in \D, \label{gconstraint}
\end{equation}
and for an upper-Galois approximation,
\begin{equation}
d \le F(q) \; \Longrightarrow \; \ell_d \preceq q \quad \forall q \in \Q_f,\; d \in \D. \label{gconstraint}
\end{equation}
{\em Closest} approximations will be had with minimal $g_d$ or maximal $\ell_d$, respectively.

\subsection{Lower-Galois enrollments for the 2-dose \mbox{3+3} trial}

\begin{lstlisting}
qs_d_nmax(Qs, D, Nmax) :- % Enumerate $q \in Q^D$ with denominators up to Nmax
  length(Qs, D),
  maplist(\Q^T^N^(Q = T/N), Qs, Ts, Ns),
  Ns ins 0..Nmax, label(Ns),
  maplist(\T^N^(T in 0..N), Ts, Ns), label(Ts).

% Qfs is the subset of $\Q_f$ with recommended dose in Range
qfs_rec(Qfs, Range) :- findall(Qf, (mendtally_rec(Qf, D), D in Range), Qfs).

g_rec(G, Rec) :- % G is a valid $g_\text{Rec}$ for the 2-dose 3+3 trial
    Rec in 0..2, indomain(Rec),
    qfs_rec(Qls, 0..Rec),
    #Rec1 #= Rec + 1,
    qfs_rec(Qhs, Rec1..2),
    qs_d_nmax(G, 2, 6),
    tfilter(\Q^(Q $\preceq$ G), Qhs, []), % no Qh falsely identified
    tpartition(\Q^(Q $\preceq$ G), Qls, _, []). % no Ql gets missed

?- setof(G0, g_rec(G0, 0), G0s).
%@    G0s = [[0/4,2/6],[1/5,0/4],[1/5,0/5],[1/5,0/6],[1/5,1/5],
%@           [1/5,1/6],[2/6,0/4],[2/6,0/5],[2/6,0/6]].

?- setof(G1, g_rec(G1, 1), G1s).
%@    G1s = [[0/6,2/6]].

?- setof(G2, g_rec(G2, 2), G2s).
%@    G2s = [[0/6,0/5],[0/6,0/6]].
\end{lstlisting}

Thus, for the 2-dose \mbox{3+3} trial design, we find (somewhat remarkably) that solutions $(g_0, g_1, g_2)$ exist for the upper adjoint $G$ even without extending our basic preorder $\preceq$.  Indeed, we may have obtained as many as 18 distinct trial designs as a result, with properties worth comparing.

One reason this result is of interest is that it may provide a formal and general basis for solving the problem of `rolling enrollment' \cite{skolnik_clinical_2007} for trial designs conceived in idealized terms which do not take account of practical realities such as unsynchronized arrivals of eligible trial participants, and pending toxicity evaluations \cite{frankel_model_2020}.

\bibliographystyle{unsrt}
\bibliography{DTAT}

\end{document}
